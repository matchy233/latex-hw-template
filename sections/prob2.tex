\begin{homeworkProblem}

  Given the following matrix
  \[
    A = \begin{bmatrix}
      2 & 1 & 0 & 0\\
      0 & 2 & 1 & 0\\
      0 & 0 & 3 & 0\\
      0 & 1 & -1 & 3
    \end{bmatrix}
  \]

  find a basis for the generalized eigenspace of $L_A$ consisting of a union of disjoint cycles of generalized eigenvectors. Then find a Jordan canonical form $J$ of $A$.

\solution

We first find the eigenvalues of $A$. Let
\[
  \det(A -\lambda I ) = 0
\]
We have

\[
  \begin{aligned}
    &\lambda^2 -10\lambda^3 + 37 \lambda^2 - 60 \lambda +36 = 0\\
    &(\lambda -3)^2(\lambda-2)^2 = 0
  \end{aligned}
\]

We can compute that $\lambda_1 = 3$ with the multiplicity of $2$, $\lambda_2 = 2$ with multiplicity of $2$, thus $\dim(K_{\lambda_1}) = 2$ and $\dim(K_{\lambda_2}) 2$. Also note that $\dim(E_{\lambda_1}) = 1 $ and $\dim(E_{\lambda_2}) = 1$, which means that the basis for $K_{\lambda_1}$ is a single cycle of length $2$ and the basis for $K_{\lambda_2}$ is a union of two cycles of length $1$.
We need to find a vector $v$ such that
\[
  \begin{aligned}
    (A-2I)v \neq 0 &\quad (A-2I)^2 v = 0\\
  (A-2I)^2 &= \begin{bmatrix}
    0 & 0 & 1 & 0\\
    0 & 0 & 1 & 0\\
    0 & 0 & 1 & 0\\
    0 & 1 & -1 & 1
  \end{bmatrix}
  \end{aligned}
\]

Basis of the solution space $(A-2I)^2 x = 0$ is \[
  \left\{
    \begin{bmatrix}
      1\\0\\0\\0
    \end{bmatrix},
    \begin{bmatrix}
      0\\-1\\0\\1
    \end{bmatrix}
  \right\}
\]
Now from that basis we choose a vector such that $(A-2I)v \neq 0$:
\[
  (A-2I)\begin{bmatrix}
    0\\-1\\0\\1
  \end{bmatrix} = \begin{bmatrix}
    -1\\0\\0\\0
  \end{bmatrix}
\]
which means that \[
  \beta_1 = \left\{
  \begin{bmatrix}
    -1\\0\\0\\0
  \end{bmatrix}, \begin{bmatrix}
    0\\-1\\0\\1
  \end{bmatrix}
  \right\}
\]
is a basis for $K_{\lambda_1}$.

Now choose eigenvectors for $A$ corresponding to eigenvalue $3$:
\[
  A - 3I = \begin{bmatrix}
    -1 & 1 & 0 & 0\\
    0 & -1 & 1 & 0\\
    0 & 0 & 0 & 0\\
    0 & 1 & -1 & 0
  \end{bmatrix}
\]

basis of the solution space of $(A-3I)x = 0$ is
\[
  \beta_2 = \left\{
  \begin{bmatrix}
    1\\1\\1\\0
  \end{bmatrix}, \begin{bmatrix}
    0\\0\\0\\1
  \end{bmatrix}
  \right\}
\]

Then the Jordan canonical basis of $A$ is \[
  \beta = \beta_1 \cup \beta_2 = \left\{
    \begin{bmatrix}
      -1\\0\\0\\0
    \end{bmatrix}, \begin{bmatrix}
      0\\-1\\0\\1
    \end{bmatrix}, \begin{bmatrix}
      1\\1\\1\\0
    \end{bmatrix}, \begin{bmatrix}
      0\\0\\0\\1
    \end{bmatrix}
  \right\}
\]
And the Jordan canonical form of $A$ is

\[
J = [A]_\beta = \begin{bmatrix}
  2 & 1 & 0 & 0\\
  0 & 2 & 0 & 0 \\
  0 & 0 & 3 & 0 \\
  0 & 0 & 0 & 3
\end{bmatrix}
\]

\end{homeworkProblem}
