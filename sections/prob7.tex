\begin{homeworkProblem}
\solution

\begin{proof}
    Let \[L_A: F^{n} \rightarrow F^{m}\] be the linear transformation coefficient matrix \[
        A = \begin{bmatrix}\vec{A}_1, \vec{A}2,..., \vec{A}n\end{bmatrix} \]
    describes and we have \[
        A\vec{x} = \vec{b}
    \]\\

<<<<<<< HEAD
    If we have a solution $\vec{s} \in F^n$ where \[
    \vec{s} = \begin{bmatrix}
        s_1 \\ s_2 \\ \vdots \\ s_n
    \end{bmatrix}
    \]  to the system,
    then \[
    \vec{b} = A\vec{s} \in R(L_A)
    \]\\
    Vice versa, if $\vec{b} \in R(L_A)$, then for $A$ there exists $\vec{s}$ where \[
        \vec{s} = \begin{bmatrix}
            s_1 \\ s_2 \\ \vdots \\ s_n
        \end{bmatrix}
        \] and \[
    b = s_1\vec{A}_1\vec + s_2\vec{A_2} + \cdots + s_n\vec{A}_n =A\vec{s}
        \]
    That means $\vec{s}$ is a solution of the system $A\vec{x} = \vec{b}$. So $A\vec{x} = \vec{b}$ has a solution if and only if $\vec{b} \in R(L_A)$.\\

    Suppose $rank(A) = m$, which means $R(L_A) \subseteq F^{m}$ and that gives $
    \vec{b} \in F^m$.
    \\
    It is known that $R(L_A) = \text{span}(\{\vec{A}1, \vec{A}2, ..., \vec{A}n\})$. Thus $A\vec{x} = \vec{b}$ has a solution if and only if $\vec{b} \in \text{span}(\{\vec{A}1, \vec{A}2, ..., \vec{A}n\})$. But $\vec{b} \in \text{span}(\{\vec{A}1, \vec{A}2, ..., \vec{A}n\})$ if and only if $\text{span}(\{\vec{A}1, \vec{A}2, ..., \vec{A}n\}) = \text{span}(\{\vec{A}1, \vec{A}2, ..., \vec{A}n, \vec{b}\})$. It is equivalent to \[
    \dim(\text{span}(\{\vec{A}1, \vec{A}2, ..., \vec{A}n\}) = \dim(\text{span}(\{\vec{A}1, \vec{A}2, ..., \vec{A}n, \vec{b}\})))
    \]
    We know that \[
    m = \dim(R(L_A)) = \dim(\text{span}(\{\vec{A}1, \vec{A}2, ..., \vec{A}n\}))
    \]
    Because $\vec{b} \in F^{m}$, it is true that \[
    m = \dim(\text{span}(\{\vec{A}1, \vec{A}2, ..., \vec{A}n, \vec{b}\}))
    \]
    So the system has a solution.
\end{proof}
=======
\solution
\begin{enumerate}[label=(\roman*)]
    \item To find the least square line $y = cx + d$, we let \[
    A = \begin{pmatrix}
        -3 & 1 \\
        -2 & 1 \\
        0 & 1 \\
        1 & 1
    \end{pmatrix}
    \quad \, \quad
    x = \begin{pmatrix}
        c \\
        d
    \end{pmatrix}
    \quad \, \quad
    y = \begin{pmatrix}
        9 \\
        6 \\
        2 \\
        1
    \end{pmatrix}
    \]Find the solution to the system by \[
    \begin{aligned}
        A^* Ax &= A^* y\\
        \begin{pmatrix}
            -3 & -2 & 0 & 1\\
            1 & 1 & 1 & 1
        \end{pmatrix}
        \begin{pmatrix}
            -3 & 1 \\
            -2 & 1 \\
            0 & 1 \\
            1 & 1
        \end{pmatrix} \begin{pmatrix} c \\ d\end{pmatrix} &= \begin{pmatrix}
            -3 & -2 & 0 & 1\\
            1 & 1 & 1 & 1
        \end{pmatrix} \begin{pmatrix}
            9 \\ 6 \\ 2 \\ 1
        \end{pmatrix}\\
        \begin{pmatrix}
            14 & -4 \\
            -4 & 4
        \end{pmatrix}\begin{pmatrix}
            c \\ d
        \end{pmatrix} &= \begin{pmatrix}
            -38 \\ 18
        \end{pmatrix}
    \end{aligned}
    \]
    Expand the computation we can solve \[
    \left\{
    \begin{aligned}
    &14c - 4d = -38 \\
    &-4c + 4d = 18
    \end{aligned}
    \right.
    \Longrightarrow
    \left\{
    \begin{aligned}
    &c = -2\\
    &d = 2.5
    \end{aligned}
    \right.
    \]
    So the linear function is \[
        y = 2x + 2.5
    \]
    and the error $E$ can be computed by
    \[
    \begin{aligned}
        E &= \norm{Ax-y}^2 \\
        &= \begin{pmatrix}
            -3 & 1 \\
            -2 & 1 \\
            0 & 1 \\
            1 & 1
        \end{pmatrix} \begin{pmatrix}
            2 \\ 2.5
        \end{pmatrix} - \begin{pmatrix}
            9 \\ 6 \\ 2 \\ 1
        \end{pmatrix}\\
        &= 1
    \end{aligned}
    \]
    \pagebreak
    \item To compute the quadratic form $y = ax + bx + c$, suppose \[
        A = \begin{pmatrix}
            9 & -3 & 1 \\
            4 & -2 & 1 \\
            0 & 0 & 1 \\
            1 & 1 & 1
        \end{pmatrix}
        \quad \, \quad
        x = \begin{pmatrix}
            a \\ b \\ c
        \end{pmatrix}
        \quad \, \quad
        y = \begin{pmatrix}
            9 \\
            6 \\
            2 \\
            1
        \end{pmatrix}
    \]
    Find the solution to the system by \[
        \begin{aligned}
            A^* Ax &= A^* y\\
            \begin{pmatrix}
                98 & -34 & 14\\
                -34 & 14 & -4\\
                14 & -4 & 4
            \end{pmatrix}\begin{pmatrix}
                a \\ b\\ c
            \end{pmatrix} &= \begin{pmatrix}
                106 \\ -38 \\ 18
            \end{pmatrix}
        \end{aligned}
        \]
        Expand the computation we can solve \[
        \left\{
        \begin{aligned}
        &98a -34b +14c = 106 \\
        &-34a + 14b -4c = -38 \\
        &14a -4b +4c = 18
        \end{aligned}
        \right.
        \Longrightarrow
        \left\{
        \begin{aligned}
        &a = \frac{1}{3}\\
        &b = -\frac{4}{3}
        &c = 2
        \end{aligned}
        \right.
        \]
        So the linear function is \[
            y = \frac{1}{3}x^2 - \frac{4}{3}x + 2
        \]
        and the error $E$ can be computed by
        \[
        \begin{aligned}
            E &= \norm{Ax-y}^2 \\
            &= \begin{pmatrix}
                9 & -3 & 1 \\
                4 & -2 & 1 \\
                0 & 0 & 1 \\
                1 & 1 & 1
            \end{pmatrix} \begin{pmatrix}
                1/3 \\ -4/3 \\ 2
            \end{pmatrix} - \begin{pmatrix}
                9 \\ 6 \\ 2 \\ 1
            \end{pmatrix}\\
            &= 0
        \end{aligned}
        \]
\end{enumerate}

>>>>>>> 3fba870 (Finish all problems)
\end{homeworkProblem}
