\begin{homeworkProblem}
\solution

\begin{proof}
    Let \[L_A: F^{n} \rightarrow F^{m}\] be the linear transformation coefficient matrix \[
        A = \begin{bmatrix}\vec{A}_1, \vec{A}2,..., \vec{A}n\end{bmatrix} \]
    describes and we have \[
        A\vec{x} = \vec{b}
    \]\\

    If we have a solution $\vec{s} \in F^n$ where \[
    \vec{s} = \begin{bmatrix}
        s_1 \\ s_2 \\ \vdots \\ s_n
    \end{bmatrix}
    \]  to the system,
    then \[
    \vec{b} = A\vec{s} \in R(L_A)
    \]\\
    Vice versa, if $\vec{b} \in R(L_A)$, then for $A$ there exists $\vec{s}$ where \[
        \vec{s} = \begin{bmatrix}
            s_1 \\ s_2 \\ \vdots \\ s_n
        \end{bmatrix}
        \] and \[
    b = s_1\vec{A}_1\vec + s_2\vec{A_2} + \cdots + s_n\vec{A}_n =A\vec{s}
        \]
    That means $\vec{s}$ is a solution of the system $A\vec{x} = \vec{b}$. So $A\vec{x} = \vec{b}$ has a solution if and only if $\vec{b} \in R(L_A)$.\\

    Suppose $rank(A) = m$, which means $R(L_A) \subseteq F^{m}$ and that gives $
    \vec{b} \in F^m$.
    \\
    It is known that $R(L_A) = \text{span}(\{\vec{A}1, \vec{A}2, ..., \vec{A}n\})$. Thus $A\vec{x} = \vec{b}$ has a solution if and only if $\vec{b} \in \text{span}(\{\vec{A}1, \vec{A}2, ..., \vec{A}n\})$. But $\vec{b} \in \text{span}(\{\vec{A}1, \vec{A}2, ..., \vec{A}n\})$ if and only if $\text{span}(\{\vec{A}1, \vec{A}2, ..., \vec{A}n\}) = \text{span}(\{\vec{A}1, \vec{A}2, ..., \vec{A}n, \vec{b}\})$. It is equivalent to \[
    \dim(\text{span}(\{\vec{A}1, \vec{A}2, ..., \vec{A}n\}) = \dim(\text{span}(\{\vec{A}1, \vec{A}2, ..., \vec{A}n, \vec{b}\})))
    \]
    We know that \[
    m = \dim(R(L_A)) = \dim(\text{span}(\{\vec{A}1, \vec{A}2, ..., \vec{A}n\}))
    \]
    Because $\vec{b} \in F^{m}$, it is true that \[
    m = \dim(\text{span}(\{\vec{A}1, \vec{A}2, ..., \vec{A}n, \vec{b}\}))
    \]
    So the system has a solution.
\end{proof}
\end{homeworkProblem}
