\begin{homeworkProblem}

  Let $T$ be a linear operator on a finite-dimensional vector space, and let $p(t)$ be the minimal polynomial of $T$. Prove the following results.
\begin{enumerate}
  \item $T$ is invertible if and only if $p(0) = 0$.
  \item If $T$ is invertible, and $p(t) = t^n + a_{n-1}t^{n-1} + \cdots + a_1t + a_0$, Then
  \[ T^{-1} = -\frac{1}{a_0}(T^{n-1}+a_{n-1}T^{n-1}+\cdots+a_2T + a_1I)\]
\end{enumerate}

\solution

\begin{enumerate}
  \item \begin{enumerate}
    \item (Prove "if") Suppose the characteristic function of $T$ is $f(t)$, and assume $T$ is an invertible operator. Then it is one-to-one, which is equivalent to $N(T) = \{ 0 \}$.

  In other words, the image of no vector other than the zero vector by $T$ equals to the zero vector. This further implies that $\lambda = 0$ cannot be an eigenvalue of $T$, since if so the dimension of its eigenspace would be greater or equal to $1$ ($\dim[N(T)] \geq 1$), which contradicts the fact that $N(T) = \{ 0 \}$.

  Therefore, $T$ is not invertible, and $f(0) \neq 0$. Since the characteristic polynomial and the minimal polynomial have the same zeros, we get $p(0) \neq 0$. To say it in another way, if $T$ is invertible, $p(0) = 0$.
  \item (Prove "only if") Let $p(0) \neq 0$, then characteristic polynomial $f(0) \neq 0$ since they have the same zeros. Hence $0$ is not an eigenvalue of $T$. Since $T$ is a linear operator, $T(0) = 0$.

  If we assume that for any other vector $x$ and some constant $c \in \mathbb{R}$, we have $T(x) = 0$, from the linearity of $T$ it follows that $T(cx) = cT(x) = 0$, which implies $\lambda = 0$ is an eigenvalue of $T$. Correspondingly, the eigenvalue dimension is at least $1$. This contradicts the previous conclusion.

  As a conclusion, $N(T) = \{ 0 \}$, which tells us that $T$ is a one-to-one operator. According to Rank-Nullity theorem, $\dim[\rank(T)] = \dim(V)$, which is equivalent to $T$ being onto.

  As a one-to-one and onto transformation, $T$ is invertible.
  \end{enumerate}
  \item According to Cayley-Hamilton theorem, $p(T) = 0$. Also, $T$ has linearity. Then we can compute:
  \[
    \begin{aligned}
      TT^{-1} &= T\left[
        -\frac{1}{a_0}(T^{n-1}+a_{n-1}T^{n-1}+\cdots+a_2T + a_1I)
      \right]\\
      &= \left[
        -\frac{1}{a_0}(TT^{n-1}+a_{n-1}TT^{n-1}+\cdots+a_2TT + a_1TI)
      \right]\\
      &= -\frac{1}{a_0}(T^{n}+a_{n-1}T^{n}+\cdots+a_2T^2 + a_1T)\\
      &= -\frac{1}{a_0}(p(T) - a_0 I)\\
      &= \frac{1}{a_0}(0 - a_0 I)\\
      &= I
    \end{aligned}
  \]
  \[
    \begin{aligned}
      T^{-1}t &= \left[
        -\frac{1}{a_0}(T^{n-1}+a_{n-1}T^{n-1}+\cdots+a_2T + a_1I)
      \right]t\\
      &= \left[
        -\frac{1}{a_0}(T^{n-1}T+a_{n-1}T^{n-1}T+\cdots+a_2TT + a_1IT)
      \right]\\
      &= -\frac{1}{a_0}(T^{n}+a_{n-1}T^{n}+\cdots+a_2T^2 + a_1T)\\
      &= -\frac{1}{a_0}(p(T) - a_0 I)\\
      &= \frac{1}{a_0}(0 - a_0 I)\\
      &= I
    \end{aligned}
  \]
  Since we show that $TT^{-1} = T^{-1}T$, $T^{-1}$ is really the inverse of $T$.
\end{enumerate}
\end{homeworkProblem}
